\item Sea $B_1=\{(1,0,1);(1,1,0);(0,0,1)\}$ una base de $\R^3$. Si
    \begin{center}
        $P_{B_2,B_1}=\begin{pmatrix}
            1&1&2\\2&1&1\\-1&-1&1
        \end{pmatrix}$.
    \end{center}
    Hallar los vectores de la base $B_2$.
    \begin{mdframed}[style=s]
        \[P_{B_2,B_1}=\left([(1,0,1)]_{B_2}\quad[(1,1,0)]_{B_2}\quad[(0,0,1)]_{B_2}\right)\]
        Osea que\[[(1,0,1)]_{B_2}=\begin{pmatrix}
            1\\2\\-1
        \end{pmatrix}\quad[(1,1,0)]_{B_2}=\begin{pmatrix}
            1\\1\\-1
        \end{pmatrix}\quad[(0,0,1)]_{B_2}=\begin{pmatrix}
            2\\1\\1
        \end{pmatrix}\]Entonces\begin{center}
            $\begin{cases}
                (1,0,1)=1v_1+2v_2-1v_3\\
                (1,1,0)=1v_1+1v_2-1v_3\\
                (0,0,1)=2v_1+1v_2+1v_3
            \end{cases}$
        \end{center}
        siendo $B_2=\{v_1,v_2,v_3\}$. Si $v_i=(v_{ix},v_{iy},v_{iz})\quad i=1,2,3\to$
        \begin{center}
            $\begin{cases}
                1=v_{1x}+2v_{2x}-v_{3x}\\
                1=v_{1x}+v_{2x}-v_{3x}\\
                0=2v_{1x}+v_{2x}+v_{3x}\\
            \end{cases},\quad\begin{cases}
                0=v_{1y}+2v_{2y}-v_{3y}\\
                1=v_{1y}+v_{2y}-v_{3y}\\
                0=2v_{1y}+v_{2y}+v_{3y}\\
            \end{cases},\quad\begin{cases}
                1=v_{1z}+2v_{2z}-v_{3z}\\
                0=v_{1z}+v_{2z}-v_{3z}\\
                1=2v_{1z}+v_{2z}+v_{3z}\\
            \end{cases}$
        \end{center}
        De donde se obtiene que \[v_1=\begin{pmatrix}
            1/3\\1\\-1/3
        \end{pmatrix},\quad v_2=\begin{pmatrix}
            0\\-1\\1
        \end{pmatrix},\quad v_3=\begin{pmatrix}
            -2/3\\-1\\2/3
        \end{pmatrix}\]
    \end{mdframed}