\item \begin{enumerate}
        \item Hallar una base de $\C_3[x]$ que no sea la canónica.
            \begin{mdframed}[style=s]
                Propongo $B_1=\{1,x,x^2+1,x^3\}$. Para verificar que sea una base, me fijo si los polinomios son li.
                \begin{center}
                    $0=\alpha+\beta x+\gamma (x^2+1)+\delta x^3\to\begin{cases}
                        \alpha+\gamma=0\\
                        \beta=0\\
                        \gamma=0\\
                        \delta=0
                    \end{cases}\to\alpha=0$
                \end{center}
                Al tener 4 polinomios li, estos generan $\C_3[x]$ y por ende son base.
            \end{mdframed}
        \item Hallar la matriz de cambio de base de $B=\{1-x,x-x^2,x^2-x^3,x^3\}$ en la base hallada en el ítem anterior.
            \begin{mdframed}[style=s]
                \begin{center}
                    $P_{B_1,B}=\left([1-x]_{B_1}\quad[x-x^2]_{B_1}\quad[x^2-x^3]_{B_1}\quad[x^3]_{B_1}\right)$\\
                    $P_{B_1,B}=\begin{pmatrix}
                        1&1&-1&0\\-1&1&0&0\\0&-1&1&0\\0&0&-1&1
                    \end{pmatrix}$
                \end{center}
            \end{mdframed}
        \item ¿Cuáles son las coordenadas de $3x^3-x+2$ en la base $B$?
            \begin{mdframed}[style=s]
                \begin{center}
                    $P_{B,\E}=\left([1]_{B}\quad[x]_{B}\quad[x^2]_{B}\quad[x^3]_{B}\right)$\\
                    $P_{B,\E}=\begin{pmatrix}
                        1&0&0&0\\1&1&0&0\\1&1&1&0\\1&1&1&1
                    \end{pmatrix}$\\
                    $\to [3x^3-x+2]_B=P_{B,\E}[3x^3-x+2]_\E$\\
                    $\to[3x^3-x+2]_B=\begin{pmatrix}
                        1&0&0&0\\1&1&0&0\\1&1&1&0\\1&1&1&1
                    \end{pmatrix}\cdot\begin{pmatrix}
                        2\\-1\\0\\3
                    \end{pmatrix}=\begin{pmatrix}
                        2\\1\\1\\4
                    \end{pmatrix}$
                \end{center}
            \end{mdframed}
    \end{enumerate}